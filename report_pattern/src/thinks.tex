\section{Выводы}
Выполнив вторую лабораторную работу по курсу \enquote{Дискретный анализ}, я освоил ключевые аспекты работы с B-деревьями, включая их реализацию, балансировку и оптимизацию для эффективного хранения и поиска данных. На практике изучил особенности обработки регистронезависимых строковых ключей и их хеширования, а также методы работы с большими числовыми диапазонами. Важным этапом стала разработка механизма сериализации и десериализации B-дерева в компактный бинарный формат с учётом возможных ошибок ввода-вывода и проверки целостности данных. Кроме того, я углубил понимание обработки системных ошибок (нехватка памяти, отсутствие прав доступа к файлу) и научился корректно возвращать диагностические сообщения без прерывания работы программы. Всё это позволило создать отказоустойчивую и эффективную структуру данных, пригодную для использования в реальных приложениях, таких как словари и базы данных.
\pagebreak
