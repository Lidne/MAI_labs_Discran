\section{Описание}
Основная идея алгоритма решения задачи состоит в использовании B-дерева для эффективного хранения и управления словарём, где ключами являются регистронезависимые слова, а значениями — соответствующие номера.
B-дерево обеспечивает быстрый поиск, вставку и удаление элементов за счёт сбалансированной структуры и оптимизации операций ввода-вывода, что особенно важно при работе с большими объёмами данных.
Для реализации необходимо разработать B-дерево с поддержкой операций добавления, удаления и поиска, а также методов Save и Load для сохранения и загрузки словаря в/из бинарного файла.

\pagebreak

\section{Исходный код}
\begin{longtable}{|p{7.5cm}|p{7.5cm}|}
\hline
\rowcolor{lightgray}
\multicolumn{2}{|c|}{btree.cpp}\\
\hline
\texttt{string to\_lower(const string \&o)} & Преобразование входной строки к нижнему регистру.\\
\hline
\texttt{BNode(bool leaf\_)} & Конструктор узла B-дерева, устанавливает флаг листа и нулевое число ключей.\\
\hline
\texttt{void BNode::insertNotNull(const string \&k, uint64\_t v)} & Вставка пары \enquote{ключ–значение} в узел (листьевой или внутренний) без проверки корня на переполнение.\\
\hline
\texttt{bool BNode::remove(const string \&key)} & Удаление ключа из поддерева с последующей балансировкой (слияние, перераспределение).\\
\hline
\texttt{pair<bool,uint64\_t> BNode::search(const string \&k)} & Бинарный поиск ключа в узле и рекурсивный спуск в потомки при необходимости.\\
\hline
\texttt{class BTree} & Внешний интерфейс B-дерева: методы \texttt{add}, \texttt{remove}, \texttt{search}, \texttt{dump}, \texttt{load}.\\
\hline
\texttt{int main()} & Цикл чтения команд из \texttt{stdin}, разбор и выполнение операций над деревом.\\
\hline
\end{longtable}

\begin{lstlisting}[language=C++]
const int T = 64;

class BNode {
public:
    bool leaf;
    int c;
    string keys[2 * T - 1];
    uint64_t values[2 * T - 1];
    BNode* children[2 * T];

    BNode(bool leaf_);
    ~BNode();

    pair<bool, uint64_t> search(const string &k);
    void insertNotNull(const string &k, uint64_t v);
    void split(int index);
    bool remove(const string &key);
};

class BTree {
public:
    BNode* root;

    BTree();
    ~BTree();

    bool add(const string &word, uint64_t val);
    bool remove(const string &v);
    pair<bool, uint64_t> search(const string &word);
    bool dump(const string &filename, string &errmsg);
    bool load(const string &fname, string &errmsg);
};

int main();
\end{lstlisting}

\pagebreak

\section{Консоль}
\begin{alltt}
george@GEORGE-PC:/mnt/c/Users/george/Projects/MaiLabs/MAI_labs_Discran/src$ ./main
+ a 1
+ A 2
+ aaaaaaaaaaaaaaaaaaaaaaaaaaaaaaaaaaaaaaaaaaaaaaaaaaaaaaaaaaaaaaaaaaaaaaaaaaaaaaaaaaaaaaaaaaaaaaaaaaaaaaaaaaaaaaaaaaaaaaaaaaaaaaaaaaaaaaaaaaaaaaaaaaaaaaaaaaaaaaaaaaaaaaaaaaaaaaaaaaaaaaaaaaaaaaaaaaaaaaaaaaaaaaaaaaaaaaaaaaaaaaaaaaaaaaaaaaaaaaaaaaaaaaaaaaaaaaaa 18446744073709551615
aaaaaaaaaaaaaaaaaaaaaaaaaaaaaaaaaaaaaaaaaaaaaaaaaaaaaaaaaaaaaaaaaaaaaaaaaaaaaaaaaaaaaaaaaaaaaaaaaaaaaaaaaaaaaaaaaaaaaaaaaaaaaaaaaaaaaaaaaaaaaaaaaaaaaaaaaaaaaaaaaaaaaaaaaaaaaaaaaaaaaaaaaaaaaaaaaaaaaaaaaaaaaaaaaaaaaaaaaaaaaaaaaaaaaaaaaaaaaaaaaaaaaaaaaaaaaaaa
A
- A
a
OK
Exist
OK
OK: 18446744073709551615
OK: 1
OK
NoSuchWord
\end{alltt}
\pagebreak

