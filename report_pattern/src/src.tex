\section{Описание}
Основная идея алгоритма решения задачи состоит в применении поразрядной сортировки, которая позволяет упорядочивать автомобильные номера в формате A 999 BC за 
линейное время благодаря их структурированности. Для реализации необходимо обработать ключи по разрядам в порядке от младшего к старшему, начиная с цифр, 
затем букв, при этом на каждом этапе применяется сортировка подсчетом для текущего разряда. Сортировка подсчетом идеально подходит, 
поскольку разряды ключей ограничены фиксированным количеством значений (10 для цифр и 26 для каждой из буквенных групп). Тип значения,
представляющий строки фиксированной длины до 64 символов, требует предварительной нормализации путем добавления нулевых символов, 
чтобы все строки стали одинаковой длины. После нормализации строки участвуют в сортировке как привязанные к ключам данные и выводятся в соответствии с упорядоченными ключами,
 при этом нулевые символы не отображаются в результате.

\pagebreak

\section{Исходный код}
На каждой непустой строке входного файла располагается пара «ключ-значение», в которой ключ указан согласно заданию, затем следует знак табуляции и указано соответствующее значение. 

\begin{lstlisting}[language=C]
	#include <bits/stdc++.h>
using namespace std;

struct Entry {
    array<uint8_t, 6> key;
    char value[65];
};

int main() {
    ios::sync_with_stdio(false);
    cin.tie(nullptr);

    vector<Entry> entries;
    string line;

    while (getline(cin, line)) {
        if (line.empty()) continue;
        auto tab = line.find('\t');

        string key_s;
        key_s.reserve(6);
        for (char c : line.substr(0, tab)) {
            if (isalnum(static_cast<unsigned char>(c)))
                key_s.push_back(c);
        }
        if (key_s.size() != 6) continue;

        Entry e;

        auto val = line.substr(tab + 1);
        size_t len = min(val.size(), size_t(64));
        memcpy(e.value, val.data(), len);
        e.value[len] = '\0';

        for (int i = 0; i < 6; ++i) {
            char c = key_s[i];
            e.key[i] = isalpha(static_cast<unsigned char>(c)) ? c - 'A' : c - '0';
        }
        entries.push_back(e);
    }

    size_t n = entries.size();
    vector<Entry> buffer(n);
    array<int, 26> cnt;
    array<int, 26> pos_arr;

    for (int d = 5; d >= 0; --d) {
        int range = (d == 0 || d >= 4 ? 26 : 10);
        cnt.fill(0);
        for (auto &e : entries) cnt[e.key[d]]++;
        pos_arr[0] = 0;
        for (int i = 1; i < range; ++i)
            pos_arr[i] = pos_arr[i - 1] + cnt[i - 1];

        for (auto &e : entries) {
            int k = e.key[d];
            buffer[pos_arr[k]++] = e;
        }
        entries.swap(buffer);
    }

    for (auto &e : entries) {
        cout << char('A' + e.key[0]) << ' ';
        cout << int(e.key[1]) << int(e.key[2]) << int(e.key[3]) << ' ';
        cout << char('A' + e.key[4]) << char('A' + e.key[5]);
        cout << '\t' << e.value << '\n';
    }
    return 0;
}

\end{lstlisting}

\pagebreak

\section{Консоль}
\begin{alltt}
george@GEORGE-PC:/mnt/c/Users/george/Projects/MaiLabs/MAI_labs_Discran/src$ make main
g++ -O2 -std=c++20 -Wall -Wextra -o main main.cpp
george@GEORGE-PC:/mnt/c/Users/george/Projects/MaiLabs/MAI_labs_Discran/src$ ./main
A 000 AA        n399tann9nnt3ttnaaan9nann93na9t3a3t9999na3aan9antt3tn93aat3naatt
Z 999 ZZ        n399tann9nnt3ttnaaan9nann93na9t3a3t9999na3aan9antt3tn93aat3naat
A 000 AA        n399tann9nnt3ttnaaan9nann93na9t3a3t9999na3aan9antt3tn93aat3naa
Z 999 ZZ        n399tann9nnt3ttnaaan9nann93na9t3a3t9999na3aan9antt3tn93aat3na
A 000 AA        n399tann9nnt3ttnaaan9nann93na9t3a3t9999na3aan9antt3tn93aat3naatt
A 000 AA        n399tann9nnt3ttnaaan9nann93na9t3a3t9999na3aan9antt3tn93aat3naa
Z 999 ZZ        n399tann9nnt3ttnaaan9nann93na9t3a3t9999na3aan9antt3tn93aat3naat
Z 999 ZZ        n399tann9nnt3ttnaaan9nann93na9t3a3t9999na3aan9antt3tn93aat3na
\end{alltt}
\pagebreak

