\section{Тест производительности}
Тест производительности представляет из себя следующее: в B‑дерево и в std::map последовательно вставляются 1.000.000 пар «ключ–значение», затем выполняется 100.000 операций поиска случайных ключей и 100.000 операций удаления. Все измерения проводятся в одной программе, выводятся отдельно для каждой фазы.

\begin{alltt}
Benchmarking B-Tree implementation:
george@GEORGE-PC:/home/george/Projects/MaiLabs/MAI_labs_Discran/src$ g++ btree.cpp main.cpp -std=c++17 -O2 -o bench_btree
george@GEORGE-PC:/home/george/Projects/MaiLabs/MAI_labs_Discran/src$ ./bench_btree
B-Tree insertion time: 0.752341 sec
B-Tree search time:    0.048912 sec
B-Tree deletion time:  0.603127 sec

Benchmarking std::map:
george@GEORGE-PC:/home/george/Projects/MaiLabs/MAI_labs_Discran/src$ g++ main_map.cpp -std=c++17 -O2 -o bench_map
george@GEORGE-PC:/home/george/Projects/MaiLabs/MAI_labs_Discran/src$ ./bench_map
std::map insertion time: 1.127589 sec
std::map search time:    0.081204 sec
std::map deletion time:  0.898432 sec
\end{alltt}

Как видно, B‑дерево опережает `std::map` по времени вставки (0.75c против 1.13c) и удаления (0.60c против 0.90c), а также чуть быстрее при поиске (0.05c против 0.08c), что демонстрирует преимущество сбалансированной структуры при больших объёмах данных.


\pagebreak

