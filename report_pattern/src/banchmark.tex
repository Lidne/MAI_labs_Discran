\section{Тест производительности}
Тест производительности заключается в сравнении поиска образца при помощи Z-алгоритма и стандартной функции строки \texttt{std::find()} в языке C++. В данном случае не учитывается время на чтение входных данных. Основная строка состоит из 1 миллиона букв, всего примерно 200 образцов длиной от 2 до 100 символов.

\begin{alltt}
[george@GEORGE-HONOR14 src]$ g++ zblock.cpp -O2
[george@GEORGE-HONOR14 src]$ ./a.out < ../input.txt > zout.txt
[george@GEORGE-HONOR14 src]$ cat zout.txt | grep "time"
Z-algorithm search total time: 0.025122 sec
[george@GEORGE-HONOR14 src]$ g++ stdfind.cpp -O2
[george@GEORGE-HONOR14 src]$ ./a.out < ../input.txt > findout.txt
[george@GEORGE-HONOR14 src]$ cat findout.txt | grep "time"
std::find search total time: 1.374654 sec
\end{alltt}

При поиске всех 200 образцов Z-алгоритм оказался значительно быстрее стандартной функции \texttt{std::find}, так как используется эффективная обработка всей строки за линейное время относительно её длины. Полученная скорость объясняется тем, что \texttt{std::find} реализует простой перебор без предварительной обработки шаблона, в то время как Z-алгоритм учитывает префиксные совпадения и работает за $O(|text| + |pattern|)$ для каждого поиска. 

Таким образом, для массового поиска образцов Z-алгоритм рекомендуется для больших текстов.

\pagebreak

